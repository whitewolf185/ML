\section{Выводы}
В ходе выполнения лабораторной работы я освежил в памяти курс математической статистики: гистограмму, корелляцию и корреляционную матрицу 
для наборов данных. Так же я изучил библиотеку Pandas, она оказалась очень удобной для анализа данных.

Трудно было найти подходящий набор данных, который подходил бы под параметры для обучения линейных моделей. В ходе своих поисков я также пробовал
провести анализ и обучение на датасете для определения качества, но он почти весь сосоял из образцов одного класса - "непригодная" вода, из-за чего
просто предсказание, что все образцы воды "плохие" можно добиться высокой точности. Oversampling тоже представлялся сложным занятием, так как количество
признаков в датасете равнялось 20 и даже небольшое изменение одного из них приравнивалось к "плохой" воде, судя по данным из датасета.

Был проанализирован набор данных Beginner's classification dataset \cite{kaggle}, результаты получились закономерные: успех изучения нового хобби напрямую зависимостей
от возраста и заинтересованности отдельно взятого человека.
\pagebreak
